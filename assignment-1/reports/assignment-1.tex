\documentclass[12pt, letterpaper]{article}
% must use this pkg for displaying imgs
\usepackage{graphicx}
\graphicspath{ {../../imgs/} }
% pkg for links
\usepackage{hyperref}
% for codeblocks highlighting
\usepackage{listings}


\title{Analyzing Zeus, an Intriguing Trojan Botnet}
\author{Akiel Aries}
\date{September 2022}
\begin{document}
\maketitle
\begin{sloppypar}

% classification img
\includegraphics[scale=0.3]{ZEUS_CLASSIFICATION.png}


When messing with malware on your machine I figured I would either want 
to run it in a VM or make sure none of the prerequisites are enabled on
the machine itself. This is where malware written in C or C++ thrive as
languages like Python and Java need the run time and interpreter already
installed to meet that prerequisite. For this assignment I decided to
take a look at the Zeus Trojan malware pkg that initially gaining
traction in the late 2000s when used to steal info from the US DOT.
Trojans are known to be malware that behaves as legitimate programs. The
bot would go on to infecting millions of computers and spawning any
variants. The bot is aimed to target windows machines to extract banking
in formations using keystroke logging. This form of attack is a common
example of a man-in-the-middle attack. The malware pkg performs
infection via phishing schemes or drive-by downloads and infected some
of the largest companies such as Bank of America, Oracle, Cisco, Amazon,
etc. The Trojan violates just about every principle in the CIA triad.
Since confidential banking information was taken as well as money,
confidentially and integrity are both violated while availability
remains in questions. Patches to the malware offered from cybersecurity
consultants is known to eradicate it from the infected machine, however
strains of the virus were spawned remaining a constant battle.
I was able to find several versions of the "leaked" banking trojan but
check out the version I had found from here!

\url{https://github.com/touyachrist/evo-zeus}

\noindent Although not the most dentrimental part of the source code, in a repo
containing the Zeus source code this block minimally shows the entry
point to the windows core API was created and giving root permission

\begin{verbatim}

void WINAPI entryPoint(void) {
  Mem::init();
  Console::init();  
  Crypt::init();
  Core::init();
  
  CUI_DEFAULT_COMMANDLINE_HELPER;

  Core::uninit();
  Crypt::uninit();
  Console::uninit();
  Mem::uninit();
  
  CWA(kernel32, ExitProcess)(coreData.exitCode);
}

%\end{verbatim}

\noindent A similar version of this code I ran on my machine running kali linux, was 
able to give me root permissions as a regular user. 
See here (maybe it works for you):

\begin{verbatim}
int getuid(){
    return 0;
}
int geteuid(){
    return 0;
}
int getgid(){
    return 0;
}
int getegid(){
    return 0;
}
\end{verbatim}

\noindent The bot has gone through many stages and versions with earlier 
implementations of the bug containing executable files hardcoded to one 
of these files

\begin{verbatim}
ntos.exe
oembios.exe
twext.exe
\end{verbatim}

\noindent Data being stored in the folowing dirs

\begin{verbatim}
System>\wsnpoem 
System>\sysproc64
System>\twain_32
\end{verbatim}

The next iteration and stored files in a single directory later found by
security researchers, with another version storing executables in
randomly named folders in app data. The bot executes in the following stages

\begin{itemize}
\item
	Builder This malware comes in a kit usable by regular users with no
  	technical knowledge, meaning the "owner" must deploy their own
  	executables to who they wish to infect. This is done easily by using
  	the builder in the tool kit. Each build will be unique to the
  	infectee, due to some cryptographic implementations, there are uniquie
  	keys generated for the configuration file embedded into the built
  	.exe.
\item
  	Configuration Seperate from the build stage, contains the address to
  	where the sniffed data is sent to in a series of blocks enables
  	customization and hardcoding settings into the final binary. The
  	config includes these sections:
  	\begin{verbatim}
	StaticConfig 
	DynamicConfig
	KeyLogger
	WebFilters
	WebDataFilters
	WebFakes
	\end{verbatim}  
\end{itemize}

Static and Dynamic Config both are dealing with the hardcoding of
settings that eventually get executed at runtime. Here the questions of
what is the target and destination get answered. In addition to the
hardcoding of generic settings, dynamic features that imply additional
comlexity such as redirection URLs of targets and destination addresses,
URL masks, log disabling, sets of URLs performing Transaction
Authenticaion Number (TAN) harvesting, as well as URL maks that contain
corresponding HTML blocks injecting into webpages who match the
Webinjects requests. The bot is responsible for running queries. \\

\noindent \textbf {Example of the config.txt file used to initate seen here:} \\
\url {https://github.com/touyachrist/evo-zeus/blob/master/output/builder/config.txt}


\noindent \textbf {Example of webinjects.txt file used for targeting:} \\
\url {https://github.com/touyachrist/evo-zeus/blob/master/source/other/webinjects.txt}

\begin{itemize}
\item
  Execution The final executable file produced from build and configure
  (sounds like installing a C application) is finally deployed by the
  "owner" of the bug. If the .exe is produced with the same cofiguration
  and build settings the end results will usually vary in where the
  config file is stored.
\item
  Server Finally the bot is deployed on a php-based server utilized by
  an abundance of php scripts that allows the deployer to monitor their
  results! This also serves as a sort of remote access type application
  where CMDs can be issued using this stage.
\item
  Why wasn't it detected? From its initial release, the bot itself was
  made more complex and harder to detect for a number a reasons. What
  made things tricky was random naming of files to specific directories
  and in small sizes. Using checksums is monitoring bits transmitted at
  a higher rate than normal, letting professionals know of some
  potential issues. In earlier stages the bug transmitted files
  carelessly and copied them into the system dir. Later versions made
  use of ensuring dropped files were not to have the same checksum as
  the orignal.
\end{itemize}

\noindent The general function of the bug is to continuously spawn threads based
on previous ones that go around crawling the infected devices hard
drive. Doing so by embedding itself into system directories. The config file is 
written into our registers and that same thread it is executed in will attempt 
to scrape for new .exe files for config to point to.

\noindent Here is an example of how the keylogging and screens scraping takes
place by importing a hook to the API

\begin{verbatim}
user32!TranslateMessage
\end{verbatim}

\noindent If the a left click is detected, a global flag is set within another PAI
hook and another check is conducted to check if the user is visiting a
location specified in the config file. Screen captures are then only
taken when visiting what is specified by the creator. 


\noindent \textbf {This uses win32 functions that can also be seen here:} \\
\url{https://github.com/touyachrist/evo-zeus/blob/master/source/client/screenshot.cpp}

\begin{verbatim}
HDC hDC = CreateCompatibleDC(0);
HBITMAP hBmp = CreateCompatibleBitmap(GetDC(0), screen_width,
screen_height);
SelectObject(hDC, hBmp);
BitBlt(hDC, 0, 0, screen_width, screen_height, x_coordinate,
y_coordinate, SRCCOPY); 
\end{verbatim}

\noindent Lets also take a look at the differences in a small implementation of
the RC4 stream cipher in a early and more recent version

\noindent The difference being the second implementation is encrypting the config
file a 0x100 byte key at build time. An extra layer scen in v2 adds more
complexity by implementing a XOR decryption (last code block)


\begin{verbatim}
/*
 * Example of config file encryption seen in v1
 */

dataSize = size of data
dataIn = encrypted data
char b;
    for (i = 0; i < dataSize; i++) {
        dataOut[i] = 0;
    }
    for (i = 0; i < dataSize; i++) {
        b = dataIn[i];
        if ((i % 2) == 0) {
            b += 2 * i + 10;
        }
        else {
            b += 0xF9 - 2 * i;
        }
        dataOut[i] += b;
    }
\end{verbatim}

\begin{verbatim}
/*
 * A look at v1.x encryption on config files
 */

int rc4_decrypt(unsigned char *in, unsigned long size, 
    unsigned char *S, unsigned char *out) {
    int i, j, dataCount;
    i = j = dataCount = 0;
    unsigned char temp, rc4_byte;
    for (dataCount = 0; dataCount < size; dataCount++) {
        i = (i + 1) & 255;
        j = (j + S[i]) & 255;
        temp = S[j];
        S[j] = S[i];
        S[i] = temp;
        rc4_byte = S[(temp + S[j]) & 255];
        out[dataCount] = in[dataCount] ^ rc4_byte;
    }   
    return dataCount;
} 
\end{verbatim}
\begin{verbatim}
for (m = (decSize-1); m >0; m--) {
    decData[m] = decData[m]^ decData[m-1];
    }
\end{verbatim}

\section*{Further Look at RC4 Stream Cipher w/ Static Analysis in CLang + cppcheck}
The RC4 Stream Cipher was an encryption algorithm used in many Windows
systems for a variety of applications. The Zeus bot exploited this
algorithm and went under a series of patches (for example using logical
operator XOR to swap instead of int/ char method)

\noindent We will take a look at how the cipher is implemented for small scale use
(passing in a key, string, expecting a returned hash). The analysis on
this isn't very exciting and was not implemented by the attackers but
exploited by them. The errors in the RC4 algorithms are beyond what I
beleive will be caught by a static analysis tool like CLang. Running the
windows-based bug on my linux machine was troublesome with CLange so I
had decided to look at a more broad issues that contributed to the bug.
So we will be looking at a minimal reproducable problem for catching
bank statements from the system.

\noindent \textbf{rc4-v0.c}\\
\noindent ITER 1 :

\begin{verbatim}
$ ./rc4-v0 1 hello
087FAE01F8
KEY = 1
STRING = hello
HASH = 087FAE01F8
\end{verbatim}

\noindent ITER 2 :

\begin{verbatim}
$ ./rc4-v0 1 HELLO
285F8E21D8
KEY = 1
STRING = HELLO
HASH = 285F8E21D8
\end{verbatim}

\noindent ITER 3 :

\begin{verbatim}
$ ./rc4-v0 224 secure_software
F96FCDD6802CC1F80298D5C5439B26
KEY = 224
STRING = secure_software
HASH = F96FCDD6802CC1F80298D5C5439B26
\end{verbatim}

\noindent ITER 4 :

\begin{verbatim}
$ ./rc4-v0 int too
28BEF0
KEY = int
STRING = too
HASH = 28BEF0
\end{verbatim}

\noindent Produces different hash based on case-type

\noindent \textbf{Static Analysis}

\begin{verbatim}
$ clang --analyze rc4-v0.c
rc4-v0.c:72:33: warning: Result of 'malloc' is converted to a pointer
of type 'unsigned char', which is incompatible with sizeof operand type
int' [unix.MallocSizeof]
    unsigned char *ciphertext = malloc(sizeof(int) * strlen(argv[2]));
    ~~~~~~~~~~~~~~~             ^~~~~~ ~~~~~~~~~~~

rc4-v0.c:79:12: warning: Potential leak of memory pointed to by
'ciphertext' [unix.Malloc]
    return 0;
           ^
\end{verbatim}






\end{sloppypar}
\end{document}
